\documentclass[handout]{beamer}
\usepackage{graphicx}

\title{Thoughts on Perturb-Seq data analysis approaches and pipelines}
\date{2025-04-29}
\author{Kirill Tsukanov \\ Senior Full Stack Developer \& Data Engineer \\ \texttt{ktsukanov@ebi.ac.uk}}
\institute{Perturbation Catalogue WP1/2/3 Technical Meeting}

\setbeamertemplate{navigation symbols}{}
\setbeamertemplate{footline}{%
  \begin{beamercolorbox}[wd=\paperwidth,ht=2.5ex,dp=1.5ex,center]{}
    \hspace{15pt}\usebeamerfont{author in head/foot}Thoughts on Perturb-Seq data analysis approaches and pipelines
    \hfill
    \usebeamerfont{date in head/foot}\insertframenumber{} / \inserttotalframenumber
    \hspace{15pt}
  \end{beamercolorbox}%
}

\begin{document}

\begin{frame}
    \titlepage
\end{frame}

\begin{frame}{Perturb-Seq data in the context of the Perturbation Catalogue}
    \begin{itemize}
        \item For MAVE and CRISPR assays, we are lucky to have curated repositories (MaveDB, DepMap) with good quality, highly processed datasets.
        \item Perturb-Seq experiments are more complex: repositories like scPerturb aggregate dozens of studies but provide mostly raw expression counts.
        \item Interpretation and downstream analysis is left to the user, which can be quite complex.
    \end{itemize}
\end{frame}

\begin{frame}{Perturb-Seq data essence}
    \begin{itemize}
        \item Each observation is roughly: perturbing \emph{gene X} in a specific cell type/tissue under set conditions yields a given gene expression profile per cell.
        \item Data characteristics:
        \begin{itemize}
            \item High noise levels;
            \item Pronounced batch effects;
            \item Large scale: thousands of cells per perturbation, multiple conditions;
            \item Raw counts require normalization, filtering, summarization, enrichment/differential expression/etc.
        \end{itemize}
        \item Raw Perturb-Seq matrices need some systematic processing to become useful for Perturbation Catalogue users.
    \end{itemize}
\end{frame}

\begin{frame}{Possible processing approaches for Perturb-Seq data}
    \begin{itemize}
        \item Specialized pipelines exist for rigorous statistical analysis:
        \begin{itemize}
            \item \textbf{Python:} MIMOSCA, MAESTRO (with partial AnnData compatibility).
            \item \textbf{R:} SCEPTRE, Mixscape.
        \end{itemize}
        \item Challenges:
        \begin{itemize}
            \item Tools are highly specialized, may require steep learning curves.
            \item For many, limited maintenance past the initial publication.
            \item Poor compatibility with the broader Python ecosystem for single cell analysis.
        \end{itemize}
    \end{itemize}
\end{frame}

\begin{frame}{scverse ecosystem}
    \begin{itemize}
        \item We are a small team and aim to deliver an MVP fast; diving into deep technical pipelines may slow progress.
        \item \textbf{scverse} offers a unified, well maintained, rapidly evolving ecosystem for single-cell analysis.
        \item Specifically, the \textbf{pertpy} tool is designed to handle single-cell perturbation workflows start to end.
    \end{itemize}
    \vspace{1em}
    \begin{columns}
        \column{0.5\textwidth}
        \centering
        \includegraphics[width=0.8\linewidth]{pertpy.png}
        \column{0.5\textwidth}
        \centering
        \includegraphics[width=0.8\linewidth]{scverse.jpg}
    \end{columns}
\end{frame}

\begin{frame}{Suggested processing approach for Perturb-Seq}
    \begin{itemize}
        \item \textbf{Input source:} scPerturb harmonised + curated to the common data schema (already in progress by Aleks).
        \item \textbf{Proposed strategy:} compute pseudobulk differential expression (because simple and robust).
        \item \textbf{Workflow:}
        \begin{enumerate}
            \item Group cells by control vs. perturbation within each cell type.
            \item Aggregate counts to pseudobulk profiles.
            \item Perform differential expression using \texttt{pertpy} facilities.
        \end{enumerate}
        \item \textbf{User-facing results:}
        \begin{itemize}
            \item ``Perturbing gene X induces significant changes in genes Y, Z...''
            \item ``Expression of gene A is most strongly altered by perturbations in genes B, C...''
        \end{itemize}
        \item Any other ideas?
    \end{itemize}
\end{frame}

\end{document}

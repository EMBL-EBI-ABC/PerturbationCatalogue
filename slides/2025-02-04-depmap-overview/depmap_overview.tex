\documentclass[handout]{beamer}
\usepackage{graphicx}

\title{Highlights on DepMap Gene Dependency Data Ingestion}
\date{2025-02-04}
\author{Kirill Tsukanov \\ Senior Full Stack Developer \& Data Engineer \\ \texttt{ktsukanov@ebi.ac.uk}}
\institute{Perturbation Catalogue WP1/2/3 Technical Meeting}

\setbeamertemplate{navigation symbols}{}
\setbeamertemplate{footline}{%
  \begin{beamercolorbox}[wd=\paperwidth,ht=2.5ex,dp=1.5ex,center]{}
    \hspace{15pt}\usebeamerfont{author in head/foot}Highlights on DepMap Gene Dependency Data Ingestion 
    \hfill
    \usebeamerfont{date in head/foot}\insertframenumber{} / \inserttotalframenumber
    \hspace{15pt}
  \end{beamercolorbox}%
}

\begin{document}

\begin{frame}
    \titlepage
\end{frame}

\begin{frame}{DepMap Data Model}
    \centering
    \includegraphics[width=0.9\linewidth]{depmap_data_model.png}
\end{frame}

\begin{frame}{DepMap Data Overview}
    \begin{itemize}
        \item 1,178 cancer screens \(\times\) 17,916 genes
        \item Data consists of CRISPR knock-out screens to identify which gene disruptions slow the growth of specific cancer cell lines.
        \item Should we use effect size or probability?
        \begin{itemize}
            \item Effect size provides granular information on the impact of gene knockout on cancer cell suppression.
            \item However, DepMap recommends using \textbf{probabilities} as they incorporate \textbf{screen quality}, unlike effect sizes.
        \end{itemize}
    \end{itemize}
\end{frame}

\begin{frame}{Missing Values}
    \begin{itemize}
        \item 829 out of 17,916 genes (4.6\%) have substantial missing data (absent in 5-30\% of screens, possibly due to batch effects).
        \item Removing these genes results in a complete dataset of 1,178 screens \(\times\) 17,087 genes, with no missing values.
    \end{itemize}
\end{frame}

\begin{frame}{Common Essential Genes}
    \begin{itemize}
        \item Some genes exhibit broad dependency patterns—knocking them out suppresses nearly all cancer lines (and likely healthy cells too!).
        \item Various filtering approaches:
        \begin{itemize}
            \item Clustering
            \item Literature on common essential genes
            \item Two datasets from DepMap
        \end{itemize}
        \item Currently using \texttt{CRISPRInferredCommonEssentials.csv}, though this may change.
        \item This reduces the dataset to 15,633 genes.
    \end{itemize}
\end{frame}

\begin{frame}{Threshold Selection}
    \begin{itemize}
        \item Probabilities cannot be normalized or converted into Z-scores—hence, we need a reasonable threshold.
        \item Optimizing for cancer cell line suppression and number of gene hits, \textbf{95\%} appears suitable.
        \item This threshold and the essential gene list will be user-configurable.
    \end{itemize}
    \begin{columns}
        \column{0.5\textwidth}
        \centering
        \includegraphics[width=\linewidth]{percentage_cells_killed.png}
        \column{0.5\textwidth}
        \centering
        \includegraphics[width=\linewidth]{number_of_hits.png}
    \end{columns}
\end{frame}

\begin{frame}{Validation for a Known Subset of Highly Essential Genes}
    \centering
    \includegraphics[width=0.9\linewidth]{validation_housekeeping_genes.png}
\end{frame}

\begin{frame}{Current Status and Next Steps}
    \begin{itemize}
        \item Data processed and ingested into Elastic.
        \item Back-end under development.
        \item Front-end concept:
        \begin{itemize}
            \item Users search for a cancer cell line of interest.
            \item Highly dependent genes for the selected cell line are displayed.
            \item Genes with associated MaveDB functional information are highlighted, enabling users to connect phenotype to gene and variant analysis.
        \end{itemize}
    \end{itemize}
\end{frame}

\end{document}
